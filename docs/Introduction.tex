\newpage
\addcontentsline{toc}{section}{Introduction}
\section*{Introduction}

Agile Wing Integration (AWI) is an interactive environment for nonlinear aeroelastic analysis and preliminary sizing of conventional and unconventional aircraft configurations. The software is written entirely in object-orientated MATLAB code and requires a valid installation of MATLAB (version 2015b or later) to be executed. 

\parindent=0pt The framework comprises of several modules which are broadly split into three functional areas:


\begin{itemize}

\item \textbf{Importing Data} - The AWI framework supports the import of data from various file types. Information about the model and/or results data can be provided in Excel format (.xlsx), Comma-Separated-Variable (.csv), Extensible Markup Language (.xml) as well as MATLAB’s proprietary file format (.mat). AWI also supports the import of Airbus aircraft models and aeroelastic/sizing data from a FAME file (.fm4) as well as models designed using the CPACS parameterisation scheme. Further information on the data import process can be found in Section...

\item \textbf{Analysing Models} - The AWI framework supports various types of aeroelastic analysis, including:

\begin{itemize}

\item \textbf{Static Aeroelastic Trim} - Including automatic calculation of the optimum control surface deflection for a given trim case. 

\item \textbf{Dynamic Aeroelasticity} - Including ‘1-minus cosine’ and continuous turbulence analysis. 

%\item \textbf{Aeroelastic Stability Analysis} – Including classical flutter analysis and nonlinear continuation %analysis. 

\item \textbf{Additional Analyses} - Further analysis types are available within AWI, including the calculation of mass and inertia properties.

\end{itemize}

\item \textbf{Visualising and Exporting Data} - The AWI framework can also be used as a means of viewing and/or exporting results and model data.  Various methods are available for viewing results quantities, including: line plots, quiver plots, deformations and animations. Furthermore, the data in an AWI session can be exported to several formats:

\end{itemize}

AWI has been developed as a research tool by staff from the Aerospace Engineering Department at the University of Bristol (UK) in collaboration with Mr Philip Rottier of MathWorks Consulting Services. There are two intended user groups for the AWI framework:

\begin{itemize}
\item \textbf{University researchers}
\item \textbf{Aerospace engineers}
\end{itemize}